\chapter{Conclusão e Trabalhos Futuros}

\noindent Neste capítulo é apresentado as conclusões obtidas através da execução deste trabalho, sendo apresentado os principais ganhos obtidos com a implantação do sistema, as sugestões para pesquisas futuras e os resultados adicionais obtidos pela implantação do sistema ``\acrshort{uft} Serviços''.

Como pode ser observado no Capítulo Resultados o sistema ``\acrshort{uft} Serviços'' apresenta um bom nível de usabilidade, desempenho e com implementações de medidas de segurança que visam manter a privacidade dos usuários do sistema.

Após a implantação do ``\acrshort{uft} Serviços'', foi constatado que o atendimento dos principais itens do levantamento de requisitos presentes na Especificação de Requisitos de \textit{Software} (IEEE 830 - Apêndice I) foram atendidos de forma satisfatória. Com a ressalva que foram realizadas algumas modificações necessárias, através da adição de novos requisitos e remoção de outros em que não seria possível o seu atendimento em um tempo hábil.

A partir da implantação do sistema, foi possível viabilizar o acesso de um aplicativo móvel pela comunidade acadêmica da \acrshort{uft} do campus de Palmas para a abertura e acompanhamento de chamados de forma acessível. Possibilitando a criação de chamados de forma detalhada, através do fornecimento de informações como a localização, foto e descrições em detalhes do problema. Sendo que o principal objetivo deste trabalho foi alcançado, pois com a implantação do sistema foi possibilitou um canal de comunicação direto com os setores responsáveis pelo atendimento da demanda, proporcionando o acompanhamento do andamento das ordens de serviços por toda a comunidade acadêmica do campus.

A aplicação do \acrshort{itil} v3 como base para o gerenciamento da implantação do sistema foi bastante produtivo, pois foi a partir da aplicação das boas práticas para gerenciamento de serviços de \acrshort{ti} presentes no ciclo de vida de serviços do \acrshort{itil} foi possível obter bons resultados quanto a gestão dos serviços oferecidos pelo \textit{software}.

A partir criação de um ambiente de integração contínua proporcionou uma mecanismo automatizado para a execução de testes, compilação (\textit{build}) e atualização do sistema \textit{deploy}, facilitando uma integração melhor de novas funcionalidades que o sistema pode demandar em uma futura atualização de seus requisitos.

Outro ponto a ressaltar, foi que a aplicação do \acrshort{itil} v3 em conjunto com a metodologia de desenvolvimento ágil \textit{Scrum} foram fundamentais para desenvolvimento e implantação do sistema. Pois através destes método foi possível obter uma melhor organização do processo de criação do \textit{software} e implantação de novos serviços.

\section{Sugestões para Trabalhos Futuros}

\noindent Um possível tema para a realização de uma pesquisa futura deste trabalho seria a partir dos dados gerados pelo sistema, realizar uma análise a partir da aplicação de técnicas de mineração de dados para extração de informações contidas em sua base de dados.

Uma outra sugestão para a realização de uma nova pesquisa seria através da proposta de implementação do cliente móvel para outras plataformas do mercado, como por exemplo para o sistema  \acrshort{ios} e Windows Phone. 

Outra abordagem para a expansão do sistema consistiria no mapeamento do patrimônio da instituição, a partir da utilização de código de barras ou \gls{qrcode}. Tal abordagem em conjunto com o mapeamento da base de dados do patrimônio da instituição possibilitaria o preenchimento automático do formulário de abertura do chamado, bem como a criação de um histórico referente ao número de vezes que um dado patrimônio passou pelo processo de manutenção, podendo agilizar o processo de substituição de um patrimônio.

Uma sugestão de trabalho futuro está na implementação de um sistema de recuperação de informação que consiste em salvar os dados do usuário em uma base de dados local durante momentos de instabilidade de conexão com a internet. Outra sugestão está na realização de mais testes para verificação da qualidade e segurança do \textit{software}, tais como: Teste de Disponibilidade, Teste de Recuperação, Teste de Regressão, Teste Funcional e Teste de Configuração.